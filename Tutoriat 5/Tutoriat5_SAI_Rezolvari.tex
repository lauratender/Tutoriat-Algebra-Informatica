\documentclass{article}
\usepackage[utf8]{inputenc}
\newcommand\tab[1][1cm]{\hspace*{#1}}
\DeclareRobustCommand\iff{\;\Longleftrightarrow\;}

\title{Tutoriat 5 \\
\Large }
\date{- 2 decembrie 2020 -}
\author{Savu Ioan Daniel, Tender Laura-Maria}

\usepackage{natbib}
\usepackage{graphicx}
\usepackage{url}
\usepackage{amsmath}
\usepackage{amssymb}
\setcounter{secnumdepth}{0}
\begin{document}

\maketitle
\section{Exercițiul 1}
Determinați elementele de ordin $30$ din $\mathbf{Z}_{240}$.
\subsection{Rezolvare:}
Căutam elementele $\widehat{x}$ cu ordinul $30$ în $\mathbf{Z}_{240} \iff$ $\widehat{x}^{30} = \widehat{0}$ și nu exista $k < 30$ astfel încât $\widehat{x}^{k} = \widehat{0} \iff x^{30} \equiv 0$ (mod $240$) și nu exista $k < 30$ astfel încât $x^{k} \equiv 0$ (mod $240$).
\newline
Fie $\widehat{k} \in \mathbf{Z}_n$, atunci ord($\widehat{k}$) = $\frac{n}{(n,k)}$.
\newline
Cum $\mathbf{Z}_{240}$ este un grup ciclic, relațiile de mai sus se pot rescrie $30x \equiv 0 $ (mod $240$). Astfel, căutăm elementele $\widehat{x}$ pentru care $\frac{240}{(240, x)} = 30 \iff (240, x) = 8$.
\newline
$240 = 2^4 \cdot 3 \cdot 5 \Rightarrow 8 \ | \ x$ dar $16 \nmid x, 3 \nmid x, 5 \nmid x$. 
\newline
$\widehat{x} \in \{\widehat{8}, \, \widehat{56}, \, \widehat{88}, \, \widehat{104}, \, \widehat{136}, \, \widehat{152}, \, \widehat{184}, \, \widehat{232}\}$.

\section{Exercițiul 2}
(i) Fie $G_1$, $G_2$ două grupuri și $x_1 \in G_1$, $x_2 \in G_2$ elemente de ordin finit. Arătați că ord($x_1, x_2$) = [ord($x_1$), ord($x_2$)].
\newline
(ii) Determinați ord([3], [4]) în grupul $\mathbf{Z}_{24} \times \mathbf{Z}_{36}$.
\subsection{Rezolvare:}
i) Fie $m$ = ord ($x_1$) și $n$ = ord ($x_2$). $x_1^m = e_1$ (în $G_1$) și $x_2^n = e_2$ (în $G_2$).  
\newline
Fie p = ord($x_1, x_2$), atunci ($x_1, x_2$)$^p$ = $e \iff (x_1^p, x_2^p) = (e_1, e_2) \iff m \ | \ p \text{ și } n \ | \ p \text{ și } p \text{ minim} \iff p = [m, n]$.  
\newline \newline
ii) Conformm i) ord([3], [4]) = [ord(3), ord(4)] în $\mathbf{Z}_{24}$, respectiv $\mathbf{Z}_{36}$.
\newline
ord(3) = $\frac{24}{(3, 24)} = 8$, ord(4) = $\frac{36}{(4, 36)} = 9$.
\newline
ord([3], [4]) = $8 \cdot 9 = 72$.

\section{Exercițiul 3}
Determinați morfisemele între grupurile aditive $\mathbf{Z}_m$ și $\mathbf{Z}_n$.
\subsection{Rezolvare:}
Fie $\{f: \mathbf{Z}_m \rightarrow \mathbf{Z}_n | \ f(\widehat{x} + \widehat{y}) = f(\widehat{x}) + f(\widehat{y}), f(\widehat{0}) = \widehat{0}\}$ morfismele căutate.
\newline \newline
Notăm $f(\widehat{1}) = \widehat{a}, \, a \in \mathbf{Z}_n$. Atunci f($\widehat{x}) = f(\widehat{1} + \widehat{1} + ... \widehat{1}) = x \cdot \widehat{a}, \,  x \in \mathbf{Z}_m$.
\newline \newline
f($\widehat{m}) = m \cdot \widehat{a} = \widehat{0} \Rightarrow n| m \cdot a. \text{ Notăm } (n, m) = d \Rightarrow \frac{n}{d} \ | \ a \Rightarrow a = \frac{n}{d} \cdot k$. 
\newline
În concluzie, $\widehat{a} = \{\widehat{0}, \widehat{\frac{n}{d}}, ..., \widehat{\frac{(d-1)n}{d}}\}$.
\section{Exercițiul 4}
Fie grupul abelian $G = \mathbf{Z}_{4} \times \mathbf{Z}_{10}$.

\begin{enumerate}
    \item Se consideră elementul $x = (\widehat{2}, \widehat{2})$ al lui $G$. Aflați ordinul lui $x$ și scrieți toate elementele subgrupului $H = \ <x>$.
    \item Cu ce grup este izomorf $G/H$?
    \item Aflați ordinul maxim al elementelor lui $G$ și dați un exemplu de element de ordin maxim.
    \item Determinați toate elementele de ordin $20$ din $G$.
\end{enumerate}
\emph{(Examen algebră, 26.01.2018, seria 10)}
\subsection{Rezolvare:}

\begin{enumerate}
\item Putem alfa ordinul elementului x prin relația ord(x) = cmmmc(ord($\widehat{2}$ în $\mathbf{Z_4}$), ord($\widehat{2}$ în $\mathbf{Z_{10}}$)). Totodată, ordinul este numărul de elemente din subgrupul $H = \{(\widehat{0}, \widehat{0}), 
(\widehat{0}, \widehat{2}), 
(\widehat{0}, \widehat{4}), 
(\widehat{0}, \widehat{6}), 
(\widehat{0}, \widehat{8}),
(\widehat{1}, \widehat{0}), 
(\widehat{1}, \widehat{2}), 
(\widehat{1}, \widehat{4}), 
(\widehat{1}, \widehat{6}), 
(\widehat{1}, \widehat{8})\}$.

\item Observăm că elementele din $G/H$ sunt de forma $G/H = \{
(\overline{\overline{0}},\overline{\overline{0}}), 
(\overline{\overline{0}}, \overline{\overline{1}}),
(\overline{\overline{1}}, \overline{\overline{0}}),
(\overline{\overline{1}}, \overline{\overline{1}})
\}$. Astfel, $G/H \cong Z_{2} \times Z_{2}$. Acestă relație de izomorfism se poate argumenta construind tabelele legilor pentru cele două grupuri. În general, avem că $Z_{n} / <x> \cong Z_{n/ord(x)}$

\item Cum am precizat și la punctul 1), ordinul elementului $(\widehat{x}, \widehat{y}) \in \mathbf{Z_{4} \times \mathbf{Z_{10}}}$ este cmmmc(ord(x), ord(y)). Din teorema lui Lagrange știm că ord(x) $|$ 4 și ord(y) $|$ 10. Alegând convenabil ord(x) = 4 și ord(y) = 10, avem ordinul maxim din grup, și anume cmmmc(4, 10) = 20. Un astfel de element cu ordin maxim este $(\widehat{1}, \widehat{1})$.

\item Mai intâi trebuie să determinăm toate combinațiile de numere cu cmmmc-ul 20. Acestea sunt (4, 10) și (4, 5). Elementele corespunzătoare ordinelor 4 și 10 sunt $(\widehat{1}, \widehat{1}), 
(\widehat{1}, \widehat{3}),
(\widehat{1}, \widehat{7}),
(\widehat{1}, \widehat{9}),
(\widehat{3}, \widehat{1}), 
(\widehat{3}, \widehat{3}),
(\widehat{3}, \widehat{7}),
(\widehat{3}, \widehat{9})
$. Elementele corespunzătoare sunt $(\widehat{1},\widehat{2}), 
(\widehat{1}, \widehat{4}), 
(\widehat{1}, \widehat{6}), 
(\widehat{1}, \widehat{8}),
(\widehat{3},\widehat{2}), 
(\widehat{3}, \widehat{4}), 
(\widehat{3}, \widehat{6}), 
(\widehat{3}, \widehat{8})
$. Acestea sunt toate elementele pentru care ordinul lor in G este 20.

\end{enumerate}

\end{document}
