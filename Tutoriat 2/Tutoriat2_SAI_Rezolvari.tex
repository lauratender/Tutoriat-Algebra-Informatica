\documentclass{article}
\usepackage[utf8]{inputenc}
\newcommand\tab[1][1cm]{\hspace*{#1}}
\DeclareRobustCommand\iff{\;\Longleftrightarrow\;}

\title{Tutoriat 2 - Rezolvări\\
\Large Funcții. Relații de echivalență.}
\date{- 10 noiembrie 2020 -}
\author{Savu Ioan Daniel, Tender Laura-Maria}

\usepackage{natbib}
\usepackage{graphicx}
\usepackage{url}
\usepackage{amsmath}
\usepackage{amssymb}
\setcounter{secnumdepth}{0}
\begin{document}

\maketitle

\section{Exercițiul 1}
Fie A, B două mulțimi finite cu $|A|$ = a și $|B|$ = b. Calculați:
\newline
i) Numărul funcțiilor strict crescătoare de la A la B.
\newline
ii) Numărul funcțiilor crescătoare de la A la B.

\subsection{Rezolvare:}
Ținând cont de faptul ca elementele mulțimiilor sunt comparabile, pentru ușurință, le putem înlocui cu numerele lor de ordine. Astfel, $A = \{1, 2, ... a\}$ și $B = \\ = \{1, 2, ..., b\}$
\newline
i) Fie $x_{1}$, $x_{2}$, ..., $x_{a}$ elementele mulțimii A cu $x_{1} < x_{2} < ... < x_{a}$. Fie $y_{1}$, $y_{2}$, ..., $y_{b}$ elementele mulțimii B cu $y_{1} < y_{2} < ... < y_{b}$. 
f : A $\rightarrow$ B strict crescătoare $\Rightarrow$ $f(x_{1}) <  f(x_{2}) < ... < f(x_{a})$.
\newline
Problema ne cere să găsim în câte moduri putem să alegem $f(x_{1}), f(x_{2}), ... f(x_{a})$ astfel încât să fie respectată condiția de funcțe crescătoare. Cum pentru fiecare alegere a unei mulțimi de elemente inclusă în B există o singură modalitate de a le aranja crescător, problema dată devine echivalentă cu următoarea: în câte moduri putem alege a numere din b numere. Dacă $b < a$ nu există nicio funcție strict crescătoare. Pentru $b \geq a$ răspunsul este $C^a_b$.
\flushleft
ii) Condiția de funcție crescătoare este $f(x_{1}) \leq  f(x_{2}) \leq ... \leq f(x_{a})$. Diferența față de punctul anterior provine din faptul că acum putem alege din mulțimea B același element de mai multe ori. Pentru a scăpa de această deosebire, putem să rescriem inegalitatea de mai sus în modul următor
\newline
$f(x_{1}) <  f(x_{2}) + 1 < f(x_{3}) + 2 <  ... < f(x_{a}) + a - 1$
\newline
Construim următoarea funcție $g : \{1, 2, ..., a\} \rightarrow \{1, 2, ..., b, b + 1, ..., b + a - 1\}$
\newline
$g(x_{k}) = f(x_k) + k - 1$. Problema inițială devine echivalentă astfel cu a determina numărul de funcții strict crescătoare de la $\{1, 2, ..., a\}$ la $\{1, 2, ..., b, b + 1, ..., b + a - 1\}$. Analog, acest număr este $C^a_{b + a - 1}$.


\section{Exercițiul 2}
Să se determine p $\in \, \mathbb{N}$* astfel încât funcția f : $\mathbb{N} \rightarrow \mathbb{N}$, f(n) = [$\frac{n}{p}$] + [$\frac{n+1}{p}$] + [$\frac{n+2}{p}$] să fie bijectivă.

\subsection{Rezolvare:}
f(n) = [$\frac{n}{p}$] + [$\frac{n+1}{p}$] + [$\frac{n+2}{p}$]
\newline
f(n+1) = [$\frac{n+1}{p}$] + [$\frac{n+2}{p}$] + [$\frac{n+3}{p}$]
\newline
f(n+1) - f(n) = [$\frac{n+3}{p}$] -  [$\frac{n}{p}$] $\geq 0 \ \forall n, p \Rightarrow$ funcția este crescătoare. 
\newline
Pentru p = 1, f(n) = 3n + 3. f(0) = 3 și cum funcția este crescătoare nu există x $\in \mathbb{N}$ astfel încât f(x) = 0, deci funcția nu este surjectivă.
\newline
Pentru p = 2, f(0) = 1 deci nici în acest caz funcția nu este surjectivă.
\newline
\textbf{Identitatea lui Hermite}
[a] + [$a + \frac{1}{n}$] + [a + $\frac{2}{n}$] + ... +  [$ a + \frac{n-1}{n}$] = [na] $\forall a \in \mathbb{R}, n \in \mathbb{N}$*
\newline
Pentru p = 3, alegând a = $\frac{n}{p}$, conform identității enunțate mai sus f(n) = n, care este bijectivă.
\newline
Pentru p $>$ 3 f(0) = f(1) = 0, deci funcția nu este injectivă.
\newline
Astfel f bijectivă pentru p = 3.

\section{Exercițiul 3}
Să se arate că funcția f : $\mathbb{N} \times  \mathbb{N} \rightarrow \mathbb{N}$ f(m, n) = $\frac{(m+n)(m+n+1)}{2}$ + m este o funcție bijectivă.

\subsection{Rezolvare:}
Funcția este surjectivă dacă $\forall x \in \mathbb{N} \ \exists \ m,\, n \in \mathbb{N} $ astfel încât f(m, n) = x. 
$\frac{(m+n)(m+n+1)}{2}$ + m = x
\newline
(m+n)(m+n+1) + 2m = 2x
\newline
Fie m + n = k, relația se rescrie astfel 
$k^2$ + k + 2m = 2x
\newline
Pentru orice x căutam cel mai mare k cu proprietatea că $k^2$ + k $\leq$ 2x. Atunci deducem că $(k + 1)^2$ + k + 1 $>$ 2x.
Atunci m = $\frac{2x - k^2 + k}{2} \in \mathbb{N}$, iar n = k - m. 
\newline
m $\leq$ k $\iff$ $\frac{2x - k^2 + k}{2} \leq $ k $\iff 2x - k^2 + k \leq$ 2k $\iff 2x \leq k^2 + 3k \iff 2x < k^2 + 3k + 2$ (conform alegerii k maxim).
\newline
Astfel funcția este sujectivă.
\newline
k(k+1) $\leq$ 2x $<$ (k+1)(k+2) = $k^2$ + 3k + 2.
\newline
Pentru a demonstra că funcția este injectivă vom presupune ca există un alt k' care îndeplinește egalitatea.
\newline
2x = $k'^2$ + k' + 2m
\newline
Cum k' nu e maxim, $(k' + 1)^2$ + k' + 1 $\leq$ 2x
\newline
Cum m  $\leq$ k',
$k'^2$ + 3k' + 2 = $k'^2$ + k' + 2k' + 2 $>$ $k^2$ + k' + 2m = 2x (contradicție).

\section{Exercițiul 4}
Fie A mulțimea $\{1, 2, ..., 2000\}$. Să se determine numărul relațiilor de echivalență pe A ale căror mulțimi factor sunt formate din două clase de echivalență.
\subsection{Rezolvare:}
Știm că oricărei relații de echivalență pe o mulțime îi corespunde într-un mod bijectiv o partiție a mulțimii. Trebuie să determinăm în câte moduri putem să partiționăm A în două mulțimi, B și C.
\newline 
Dacă B are 1 element, atunci avem $C^1_{2000}$ posibilități.
\newline
Dacă B are 2 elemente, atunci avem $C^1_{2000}$ posibilități.
\newline
...
\newline
Dacă B are 999 elemente, atunci avem $C^{999}_{2000}$ posibilități.
\newline
Daca B are 1000 elemente, atunci avem $\frac{C^{1000}_{2000}}{2}$ posibilități (deoarce nu contează ordinea pentru B și C, iar o alegere a elementelor lui B determina elementele lui C).
\newline
Nu vom lua cazurile pentru care B are mai mult de 1000 de elemente deoarece se regăsesc in cazurile de mai sus, B C fiind echivalent cu C B.
\newline
Folosindu-ne de prorpietățiile combinăriilor, și anume
\newline
$C^k_n = C^{n - k}_n$
\newline
$C^0_n + C^1_n + ... + C^n_n = 2^n$
rezultatul este
\newline
$N = 2^{n-1} - 1$



\end{document}
